\section{Theoretical Implications}

\subsection{Can We Solve Problems Without Knowing Algorithms?}

\textbf{Answer: YES, with important caveats.}

\paragraph{Requirements for Algorithm-Independent Solving:}
\begin{enumerate}
    \item \textbf{Evaluation oracle} $O(s)$ exists and is polynomial-time
    \item \textbf{State space} is sampleable in polynomial time
    \item \textbf{Structure} is learnable (bounded VC dimension)
\end{enumerate}

\begin{definition}[Natural-Solvable]
Problem $P$ is natural-solvable if there exists:
\begin{enumerate}
    \item Polynomial-time oracle $O_P$
    \item Polynomial-time sampler $\text{Gen}_P$
    \item Learning algorithm $\mathcal{L}$ with sample complexity $\poly(n)$
    \item Morphism $\varphi$ learned by $\mathcal{L}$ achieving $(1+\varepsilon)$-approximation
\end{enumerate}
\end{definition}

\begin{conjecture}
Natural-solvable problems form complexity class $\OT$, with $\Pclass \subseteq \OT \subseteq \NP$.
\end{conjecture}

\subsection{Can We Compute Non-Computable Functions?}

\textbf{Answer: NO, but we can approximate with provable bounds.}

\paragraph{Fundamental Limits (Turing-Church Thesis):}
\begin{itemize}
    \item Halting problem: Cannot decide if program halts
    \item Kolmogorov complexity $K(x)$: Cannot compute exactly
    \item Busy Beaver $BB(n)$: Non-computable function
\end{itemize}

\paragraph{What NOBC CAN do:}
\begin{enumerate}
    \item \textbf{Upper bounds}: $\hat{K}(x) \geq K(x)$ for Kolmogorov complexity
    \item \textbf{Probabilistic estimates}: $P(\text{halts}) \approx$ observed halt rate
    \item \textbf{Observable subsets}: Solve restricted versions
\end{enumerate}

\subsection{Complexity Class $\OT$ (Observation Time)}

\begin{definition}[$\OT$ Class - Formal]
A problem $L \in \OT$ if there exists:
\begin{enumerate}
    \item Oracle $O: S \to \mathbb{R}$ computable in $\poly(n)$ time
    \item Sampler $\text{Gen}: \{0,1\}^n \to S$ generating candidates in $\poly(n)$ time
    \item Learner $\mathcal{L}$ with sample complexity $M = \poly(n, 1/\varepsilon)$
    \item Morphism $\varphi$ learned by $\mathcal{L}$ achieving $(1+\varepsilon)$-approximation
\end{enumerate}
with total time $O(M \cdot \poly(n)) = \poly(n, 1/\varepsilon)$.
\end{definition}

\paragraph{Conjectured Relationships:}
\begin{equation}
\Pclass \subseteq \OT \subseteq \NP
\end{equation}

\paragraph{Evidence for $\OT$:}
\begin{itemize}
    \item \textbf{TSP}: 68\% optimal (large $\OT$ subset exists)
    \item \textbf{Average-case complexity}: Many NP-complete problems easy on average
    \item \textbf{Practical algorithms}: Heuristics work well in practice
    \item \textbf{Phase transitions}: Empirically observed easy/hard boundaries
\end{itemize}

\subsection{Predictability and Self-Computing Depth}

\begin{hypothesis}[Phase-Complexity Connection]
Problems with self-computing depth $d(F) \approx 2$--$3$ (Phase D) are optimal for NOBC:
\begin{itemize}
    \item $d<2$: Too simple, classical algorithms suffice
    \item $2\leq d\leq 3$: Rich structure but learnable $\Rightarrow$ NOBC optimal zone
    \item $d>3$: Too complex, structure not learnable
\end{itemize}
\end{hypothesis}

\paragraph{Evidence from TSP:}
\begin{itemize}
    \item \textbf{Euclidean instances}: $d\approx 1$--$1.5$ $\Rightarrow$ Simple patterns, multiple methods work
    \item \textbf{Market instances}: $d\approx 2$--$2.5$ $\Rightarrow$ Richer structure, NOBC has advantage
    \item \textbf{Pathological instances}: $d\approx 2.5$--$3$ $\Rightarrow$ Critical zone, only NOBC succeeds
    \item \textbf{Adversarial}: $d>3$ $\Rightarrow$ No structure, all methods struggle
\end{itemize}

\begin{table}[h]
\centering
\begin{tabular}{@{}llr@{}}
\toprule
\textbf{Phase} & \textbf{Depth $d$} & \textbf{Predictability $\Psi$} \\ \midrule
Phase B & $d\approx 1$ & $\Psi\approx 0.8$ (Christofides works) \\
Phase C & $d\approx 1.5$--$2$ & $\Psi\approx 0.6$ (SimAnneal competitive) \\
Phase D & $d\approx 2.5$--$3$ & $\Psi\approx 0.3$ (NOBC dominates) \\
Phase E & $d>3$ & $\Psi\to 0$ (All methods fail) \\ \bottomrule
\end{tabular}
\caption{Predictability and algorithm performance by phase}
\label{tab:phase_performance}
\end{table}
