\section{Introduction}

\subsection{The Algorithmic Paradox}

Classical computer science operates under a fundamental assumption: to solve a problem, one must possess an explicit algorithm that encodes the solution procedure. This paradigm has been extraordinarily successful for problems in complexity class $\Pclass$, where polynomial-time algorithms exist. However, for NP-complete problems like the Traveling Salesman Problem (TSP), no polynomial-time exact algorithm is known, and we rely on either exponential exact methods (Held-Karp: $O(2^n n^2)$) or approximate heuristics (Christofides: 1.5-approximation for metric TSP).

Yet nature routinely ``solves'' complex optimization problems without explicit algorithms:
\begin{itemize}
    \item \textbf{Physical systems} minimize free energy to find stable configurations
    \item \textbf{Biological organisms} navigate complex environments through observation and learning
    \item \textbf{Neural systems} learn patterns from examples without explicit programming
\end{itemize}

This raises a fundamental question: \textbf{Can we solve computational problems through observation and structural learning, without requiring an explicit algorithm?}

\subsection{The Natural Observation Hypothesis}

We propose that many computational problems can be solved by:

\begin{enumerate}
    \item \textbf{Sampling} the state space of possible solutions
    \item \textbf{Observing} the quality (cost/fitness) of sampled states
    \item \textbf{Learning} structural patterns that distinguish good from bad solutions
    \item \textbf{Converging} to optimal solutions through free energy minimization
\end{enumerate}

Crucially, this approach requires only:
\begin{itemize}
    \item \textbf{Evaluation oracle} $O(s)$: can assess quality of candidate solution $s$ in polynomial time
    \item \textbf{Sampleable state space}: can generate random candidates in polynomial time
    \item \textbf{Learnable structure}: problem exhibits patterns that can be captured through observation
\end{itemize}

No knowledge of the optimal algorithm is required.

\subsection{Theoretical Foundations: Symbolic DNA and Self-Computing Systems}

Our approach is grounded in the \textbf{Symbolic Structures Framework} \citep{kotikov2025}, where computational systems are viewed as \textbf{self-computing functorial objects} operating on \textbf{symbolic DNA} representations. Key principles:

\paragraph{1. Structural Representation:} Any computational problem can be encoded as a string $\sigma \in \Sigma^*$ over a finite symbolic alphabet $\Sigma$. The specific alphabet is chosen adaptively based on the structure of input data through encoder functions that map problem-specific patterns to abstract symbols. For example, for TSP, distance relationships are encoded; for market data, OHLCV patterns are translated to symbols representing structural changes (growth, decline, neutral states, phase transitions). The encoding preserves structural properties while abstracting away numerical details.

\paragraph{2. Functorial Computation:} Computation is not rule application but \textbf{morphism composition} in a category of structures:
\begin{equation}
F: \Sigma^* \to \Sigma^*
\end{equation}
where $F$ represents the self-computing functor that transforms one structural state into another.

\paragraph{3. Observation as Learning:} Instead of computing $F$ explicitly, we \textbf{learn $F$ from observations} by sampling the space and identifying structural invariants---patterns that remain stable under the functor's action.

\paragraph{4. Predictability Hierarchy:} Systems are classified by their \textbf{depth of self-computation} $d(F)$, ranging from fully deterministic ($d=0$, class A) to fractally self-reflective ($d\to\infty$, class E). We conjecture that practical NP-complete instances lie in intermediate class $\OT$ (Observation Time, $1\leq d\leq 3$), where structure is learnable but exact algorithms are intractable.

\subsection{Contributions}

\begin{enumerate}
    \item \textbf{Theoretical Framework:} We formalize Natural Observation-Based Computing within category-theoretic foundations, defining it as morphism learning in the space of symbolic structures.
    
    \item \textbf{Computational Paradigm:} We demonstrate that problems can be solved without algorithm knowledge, requiring only evaluation oracle and structural learnability.
    
    \item \textbf{Topological Discovery:} Through systematic analysis of information space topology at computational depths $d=1$ to $d=8$, we reveal:
    \begin{itemize}
        \item \textbf{Five-Dimensional Structure:} Independent cycles ($\beta_1$) peak at $d=5$ with 3,104 topological features, representing 4.08$\times$ increase from $d<5$ baseline
        \item \textbf{Universal Complexity Bound:} Dimensional collapse beyond $d=5$ (60\% reduction by $d=8$), validating theoretical bound $d_{\max} \approx \ln(N)/\ln(\sigma)$
        \item \textbf{Three-Phase Behavior:} Growth phase ($d \leq 4$), peak complexity ($d=5$), collapse phase ($d>5$) with exponential fragmentation
        \item \textbf{Tractability Limit:} Problems requiring $d>5$ self-computational depth exceed natural observation capacity, explaining practical NP-hardness
    \end{itemize}
    Statistical validation confirms all topological findings ($p<0.01$, combined $p<0.0001$).
    
    \item \textbf{Empirical Validation:} Through 300 comprehensive TSP tests, we show NOBC achieves:
    \begin{itemize}
        \item 68\% optimal solutions (vs 3\% for Christofides, $p<0.001$)
        \item 2.1\% average deviation (vs 31.8\% for Christofides, $p<0.001$)
        \item 82--84\% improvement on pathological graphs ($p<0.001$, huge effect sizes)
    \end{itemize}
    
    \item \textbf{Complexity Implications:} We propose complexity class $\OT \subseteq \NP$, characterized by polynomial observation time with learnable structure, and provide empirical evidence that 68\% of TSP instances lie in this class.
    
    \item \textbf{Statistical Rigor:} All results validated with non-parametric statistical tests (Wilcoxon, Friedman), effect size analysis (Cohen's $d$), and 95\% confidence intervals.
\end{enumerate}
