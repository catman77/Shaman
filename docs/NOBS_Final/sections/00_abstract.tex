We introduce \textbf{Natural Observation-Based Computing} (NOBC), a fundamentally new computational paradigm that solves optimization problems through direct observation and structural learning rather than algorithmic search. Unlike traditional approaches that require explicit algorithms, NOBC operates by sampling the state space, learning structural patterns through probabilistic observation, and converging to solutions via free energy minimization---analogous to natural physical processes. 

We demonstrate this paradigm's effectiveness on the Traveling Salesman Problem (TSP), achieving 68\% optimal solution rate and 2.1\% average deviation across 300 comprehensive tests, significantly outperforming classical approximation algorithms (Christofides: 31.8\% deviation) especially on pathological instances where traditional methods catastrophically fail (29--85\% deviation vs. 0--2\% for NOBC). Statistical validation confirms all results are highly significant ($p<0.001$, Cohen's $d=0.5$--$3.4$). 

We propose that NOBC represents an intermediate complexity class $\OT$ (Observation Time) between $\Pclass$ and $\NP$, offering practical solutions for NP-complete problems without exponential search. The approach is grounded in category-theoretic foundations \cite{kotikov2025} where computation is viewed as morphism learning in the space of structural representations, and the computational system itself is modeled as a self-computing functorial object encoded in symbolic DNA. Empirical validation on financial market data reveals that information space exhibits five-dimensional topological structure with a universal complexity bound at $d=5$, beyond which dimensional collapse occurs.
